The portmanteau test checks the null hypothesis that there is no remaining residual autocorrelation at lags $1$ to $h$ against the alternative that at least one of the autocorrelations is nonzero. In other words, the pair of hypotheses:
\begin{align*}
    H_0:\rho_u(1)=\rho_u(2)=...=\rho_u(h) = 0
\end{align*}
versus:
\begin{align*}
    H_1:\rho_u(j)\neq 0 \text{ for at least one }j=1,...,h
\end{align*}
is tested, where $\rho_u(j) = Corr(u_t, u_{t-j})$ denotes an autocorrelation coefficient of
the residual series. Consider the Box-Pierce test statistic $Q_h$
\begin{align*}
    Q_h = T \sum_{j=1}^h \hat{\rho}^2_u(j)
\end{align*}
which has an approximate $\chi^2(h-p)$-distribution if the null hypothesis holds and $T$ is the length of the residual series. The null hypothesis of no residual autocorrelation is rejected for large values of the test statistic.

\begin{itemize}
    \item Load Quarterly data for the price index of US Gross National Product given in \texttt{load gnpdeflator.xlsx}. This is a chain-type price index with basis year 2005. The data is seasonally adjusted and spans the period from 1954.Q4 to 2007.Q4.
    \item Compute the inflation series. That is, take the first difference of log(gnpdeflator).
    \item Use the Akaike information criteria to determine the lag length $\hat{p}$.
    \item Estimate two models: (i) an $AR(\hat{p})$ model and (ii) an $AR(1)$ model with OLS.
    \item Set $h=\hat{p}+10$ and compute $Q_h$ as well as the corresponding p-value for both models.
    \item Comment, based on your findings, whether the residuals are white noise.
\end{itemize}
\begin{solution}\textbf{Solution to \nameref{ex:Portmanteau}}
    \lstinputlisting{../progs/PortmanteauTest.m}
    The Null hypothesis of no remaining residual autocorrelation can be rejected for the $AR(1)$ model but not for the $AR(\hat{p})$ model.
\end{solution}
