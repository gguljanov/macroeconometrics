Broadly define the term and research topics of \enquote{Macroeconometrics}. How does the traditional (Cowles Commission) approach differ from the modern practice using structural vector autoregressive (SVAR) and dynamic stochastic general equilibrium (DSGE) models? What are the challenges in SVARs and DSGEs and their econometric evaluation?
\begin{solution}\textbf{Solution to \nameref{ex:Macroeconometrics}:}
    \begin{itemize}
        \item Definition of term: Combination of modern theoretical macroeconomics (the study of aggregated variables such as economic growth, unemployment and inflation by means of structural macroeconomic models) and econometric methods (the application of formal statistical methods in empirical economics).

        \item Research topics:
              \begin{itemize}
                  \item How to identify sources of fluctuations, e.g. how important are monetary policy shocks as opposed to other shocks for movements in aggregate output? [forecast error variance decomposition]

                  \item Understand propagation of shocks, e.g. what happens to aggregate hours worked over the next two years in response to a technology shock in the current quarter? [impulse response function]

                  \item Forecasting the future, e.g. how will inflation and output growth rates evolve over next eight quarters. [forecasting]

                  \item Predict effect of policy changes, e.g. how will output and inflation respond to an unanticipated change in nominal interest rate? [impulse response function and forecast scenarios]

                  \item Structural changes in the economy, e.g. has monetary policy changed in the early 1980s, why did volatility of many macroeconomic time series drop in the mid 1980s, [historical decomposition]

                  \item How much of the recession of 1982  would have deepened had monetary policymakers not responded to output growth at all. [policy counterfactual]
              \end{itemize}

        \item Traditional Cowles Commission approach: large-scale system-of-equations models developed from 1950s to 1970s. Came under attack in mid 1970s by (i) Lucas (1976) who argued that these models are unreliable tools for policy analysis because they are unable to predict effects of policy regime changes on the expectation formation of economic agents in a coherent manner $\rightarrow$ development of micro-founded dynamic stochastic general equilibrium (DSGE) models (RBC, new-Keynesian) and (ii) Sims (1980) who criticized that many restrictions that are used to identify behavioral equations in these models are inconsistent with dynamic macroeconomic theories $\rightarrow$ (structural) vector autoregressions ((S)VAR) as alternative.

        \item SVAR approach: multivariate linear representation of a vector of observables on its own lags and (possibly) other variables as a trend or a constant. SVARs make explicit identifying assumptions to isolate estimates of policy and/or private agent's behavior and its effects on the economy while keeping the model free of many additional restrictive assumptions needed to give every parameter a behavioral interpretations. Identification restrictions, however, are dictated by the economic theory being studied.

        \item DSGE approach: decision rules of economic agents are derived from assumptions about preferences and technologies by solving intertemporal optimization problems. Agents face uncertainty with respect to e.g. total factor productivity or nominal interest rate set by central bank. Uncertainty is generated by exogenous stochastic processes or shocks that shift technology or generate unanticipated deviations from a central bank's interest-rate feedback rule. Conditional on distributional assumptions for the exogenous shocks, the DSGE model generates a joint probability distribution for the endogenous model variables such as output, consumption, investment, and inflation.

        \item Challenges in SVAR and DSGE:
              \begin{itemize}
                  \item SVAR: many free parameters to estimate, need restrictions for e.g. good forecasts, difficult to predict effects of (rare) policy regime changes as expectation formation and behavior of economic agents are not explicitly modeled

                  \item DSGE: trade-off between theoretical coherence and empirical fit

                  \item Model specification: How to select the \enquote{right} model, how to cope with misspecification of SVARs and DSGE when taking them to data?

                  \item Identification: parameters of a model are identifiable if no two parameterizations (or structures) of that model generate the same probability distribution for the observables. Problem of identification: flat likelihood estimation and inference not possible. In SVARs we need additional restrictions to identify structural, e.g. a monetary policy, shocks as the mapping between one-step-ahead forecast errors (residuals of reduced-form VAR) and the structural shocks is not unique. Do we want these restrictions to be compatible with a large class of models or find the one concrete and \enquote{correct} model? DSGE models can generate many parameter identification problems inherent in e.g. solution techniques as teh mapping from the structural parameters into the (approximated) solution parameters is highly nonlinear and typically can only be evaluated numerically.

                  \item Macroeconomic time series are indicative of nonlinearities and time-varying volatility. In (S)VARs nonlinear dynamics are typically generated with time-varying coefficients, whereas most DSGE models are nonlinear and only for convenience approximated by linear rational expectations models. Implementation of computations is in both cases often cumbersome and challenging.
              \end{itemize}
    \end{itemize}
\end{solution}
