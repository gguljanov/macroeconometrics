Three important fundamental concepts when using time series models are the white-noise process, stationarity and lag-operator.
\begin{solution}\textbf{Solution to \nameref{ex:FundamentalConceptsUnivariateTSA}}
\end{solution}

\begin{enumerate}
    \item Define a White Noise process $WN(0,\sigma_\varepsilon^2)$. Plot 200 observations of
          \begin{align*} (i)~y_t & =\varepsilon_t \\ (ii)~y_t &= \frac{1}{5}(\varepsilon_{t-2}+\varepsilon_{t-1}+\varepsilon_{t}+\varepsilon_{t+1}+\varepsilon_{t+2})
          \end{align*} with $\varepsilon_{t} \sim N(0,1)$. What are the differences?
          \begin{solution}
              A white noise has mean zero and constant variance $\sigma^2$, whereas all other autocovariances are zero, i.e. $Cov(\varepsilon_{t},\varepsilon_s) = E(\varepsilon_t \varepsilon_s) - E(\varepsilon_t)E(\varepsilon_s) = E(\varepsilon_t \varepsilon_s) = 0$ for $s \neq t$.
              \lstinputlisting{../progs/WhiteNoisePlots.m}
              Every simulation is different, model can thus generate an infinite set of realizations over the period $t=1,...,200$.

              (i) is a white noise process, it is not persistent.
              (ii), the 5-point-moving-average is more in line with real observations, as it is a linear combination of white noise processes. It is smoother and more persistent and very different from just noise. Linear combinations of white noise processes build the fundament of many models in time series analysis.
          \end{solution}

    \item Briefly explain the concepts of weak and strict stationarity. Define the autocovariance and autocorrelation function for a stationary stochastic process $\{Y_t\}$.
          \begin{solution}
              Weak stationarity or covariance stationarity: $E(Y_t)=\mu$ (usually $\mu=0$), $var(Y_t)=E(Y_t - \mu)(Y_t-\mu)=\sigma_Y^2$ for all $t$ and $E(Y_t-\mu)(Y_s-\mu)=E(Y_t-\mu)(Y_{t-|t-s|}-\mu)=0$ for $t\neq s$, i.e. no autocorrelation.\\
              Strict stationarity: for all $m$ and $h$ $$f(Y_t,Y_{t-1},...,Y_{t-m})=f(Y_{t-h},Y_{t-h-1},...,Y_{t-h-m})$$
              Autocovariance function for a stationary process: $$\gamma_k = E(Y_t - \mu)(Y_{t-k}-\mu)$$ where $\gamma_0$ is the variance. Autocorrelation function: $$\rho_k = \gamma_k/\gamma_0$$ How to estimate:
              \begin{align*}
                  \hat{\gamma}_k = c_k & = \frac{1}{T} \sum_{y_t -\bar{y}}(y_{t-k}-\bar{y}) \\
                  \hat{\rho}_k = r_k   & = c_k/c_0
              \end{align*}
              Note: Degrees of freedom are sometimes corrected by the observations needed to estimate the parameters of the model.
          \end{solution}

    \item Consider the linear first-order difference equation $$y_t=\phi y_{t-1}+\varepsilon_t$$ Simulate and plot 200 observations of (i) $|\phi|<1$, (ii) $\phi=1$ and (iii) $|\phi| >1$. What does this suggest in terms of stationarity?
          \begin{solution}~
              \lstinputlisting{../progs/ARPlots.m}
              Remarks: If $|\phi|<1$ the series returns to the mean, i.e. it is stable and stationary. If $|\phi>1|$ then it explodes, i.e. it is unstable and not stationary. $\phi=1$ is a so-called random walk, it is the key model when working with non-stationary models. Note that the random walk incorporates many different shapes, in macroeconomic forecasts we often want to \enquote{beat} the random walk model.
          \end{solution}

    \item Briefly explain the Lag-operator and Lag-polynomials. How do you check stationarity?
          \begin{solution}
              It is a special LINEAR operator, similar to the expectation operator, and very useful when working with time series. The operator transforms one time series into another by shifting the observation from period $t$ to period $t-1$: $Ly_t = y_{t-1}$ or $L^{-1} y_t =y_{t+1}$. More general: $L^k y_t = L^{k-1} L y_t = L^{k-1} y_{t-1} = ... = y_{t-k}$. Convenient use:
              $$(1-L)y_t = y_t - y_{t-1}= \Delta y_t$$

              We can also work with lag-polynomials:$$ \phi(L) = (1-\phi_1 L-\phi_2 L^2 -... - \phi_p L^p)$$ where we call p the lag order. So:
              $$ \phi(L) y_t = (1-\phi_1 L-\phi_2 L^2 -... - \phi_p L^p)y_t = y_t - \phi_1 y_{t-1} -\phi_2 y_{t-2} - ... - \phi_p y_{t-p}$$
              For stationarity: check whether the roots of the lag-polynomial (treat $L$ as a complex number $z$) lie outside the unit circle.
          \end{solution}
\end{enumerate}
