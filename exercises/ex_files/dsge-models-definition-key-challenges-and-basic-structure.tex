\begin{enumerate}
    \item Briefly define the term and key challenges of Dynamic Stochastic General Equilibrium (DSGE) models. What are DSGE models useful for?
          \begin{solution}
              DSGE models use modern macroeconomic theory to explain and predict co-movements of aggregate time series. DSGE models start from what we call the micro-foundations of macroeconomics (i.e. to be consistent with the underlying behavior of economic agents), with a heart based on the rational expectation forward-looking economic behavior of agents. In reality all macro variables are related to each other, either directly or indirectly, so there is no \enquote{cetribus paribus}, but a dynamic stochastic general equilibrium system.
              \begin{itemize}
                  \item General Equilibrium (GE): equations must always hold. Short-run: decisions, quantities and prices adjust such that equations are full-filled. Long-run: steady state, i.e. a condition or situation where variables do not change their value (e.g. balanced-growth path where the rate of growth is constant).
                  \item Stochastic (S): disturbances (or shocks) make the system deviate from its steady state, we get business cycles or, more general, a data-generating process
                  \item Dynamic (D): Agents are forward-looking and solve intertemporal optimization problems. When a disturbance hits the economy, macroeconomic variables do not return to equilibrium instantaneously, but change very slowly over time, producing complex reactions. Furthermore, some decisions like investment or saving only make sense in a dynamic context. We can analyze and quantify the effects after (i) a temporary shock: how does the economy return to its steady state, or (ii) a permanent shock: how does the economy move to a new steady state.
              \end{itemize}
              Basic structure:
              $$ E_t \left[f(y_{t+1}, y_t, y_{t-1},u_t)\right]=0$$
              where $E_t$ is the expectation operator with information conditional up to and including period $t$, $y_t$ is a vector of endogenous variables at time $t$, $u_t$ a vector of exogenous shocks or random disturbances with proper density functions. $f(\cdot)$ is what we call economic theory. \textbf{First key challenge:} value of endogenous variables in a given period of time depend on its future expected value. We need dynamic programming techniques to find the optimality conditions which define the economic behavior of the agents.

              The solution to this system is a decision function:
              $$y_t = g(y_{t-1},u_t)$$ \textbf{Second key challenge}: DSGE models cannot be solved analytically, except for some very simple and unrealistic examples. We have to resort to numerical methods and a computer to find an approximated solution.

              Once the theoretical model and solution is at hands, the next step is the \textbf{third key challenge}: application to the data. The usual procedure consists in the calibration of the parameters of the model using previous information or matching some key ratios or moments provided by the data, or more recently, form the estimation of the parameters using maximum likelihood, Bayesian techniques, indirect inference, or general method of moments.


          \end{solution}
    \item Outline the common structure of a DSGE model. How do the neoclassical, New-Classical and New-Keynesian models differ?
          \begin{solution}
              Focus on behavior of three main types of economic agents or sectors:
              \begin{itemize}
                  \item Households: benefit from private consumption, leisure and possibly other things like money holdings or state services; subject to a budget constraint in which they finance their expenditures via (utility-reducing) work, renting capital and buying (government) bonds $\hookrightarrow$ maximization of utility
                  \item Firms produce a variety of products with the help of rented equipment (capital) and labor. They (possibly) have market power over their product and are responsible for the design, manufacture and price of their products. $\hookrightarrow$
                        cost minimization or profit maximization
                  \item Monetary policy follows a feedback rule, so-called Taylor rule, for instance: nominal interest rate reacts to deviations of the current (or lagged) inflation rate from its target and of current output from potential output
                  \item Fiscal policy (the government) collects taxes from households and companies in order to finance government expenditures (possibly utility-enhancing) and government investment (possibly productivity-enhancing). In addition, the government can issue debt securities.
              \end{itemize}
              Also other sectors possible: financial sector, foreign sector, etc. Equilibrium results from the combination of economic decisions taken by all economic agents.

              \begin{itemize}
                  \item Canonical neoclassical model (RBC model): reduce economy to the interaction of just one (representative) consumer/household and one (representative) firm. Representative household takes decisions in terms of how much to consume (save) and how much time is devoted to work (leisure). Representative firm decides how much it will produce. Equilibrium of the economy will be defined by a situation in which all decisions taken by all economic agents are compatible and feasible. One can show that business cycles can be generated by one special disturbance: total factor productivity or neutral technological shock; hence, model generates real business cycles without nominal frictions.
                  \item Scale of DSGE models has grown over time with incorporation of a large number of features. To name a few: consumption habit formation, nominal and real rigidities, non-Ricardian agents, investment adjustment costs, investment-specific technological change, taxes, public spending, public capital, human capital, household production, imperfect competition, monetary union, steady state unemployment etc.
                  \item New-Keynesian models have the same foundations as New-Classical general equilibrium models, but incorporate different types of rigidities in the economy. Whereas new classical DSGE models are constructed on the basis of a perfect competition environment, New-Keynesian models include additional elements to the basic model such as imperfect competitions, existence of adjustment costs in investment process, liquidity constraints or rigidities in the determination of prices and wages.
              \end{itemize}
          \end{solution}
    \item Comment whether or not the assumptions underlying DSGE models should be as realistic as possible. For example, a most-common assumption is that all agents live forever.
          \begin{solution}
              The degree of realism offered by an economic model is not a goal to be pursued by macroeconomists, but rather the model's usefulness in explaining macroeconomic reality. General strategy is the construction of formal structures through equations that reflect the interrelationships between the different economic variables. These simplified structures is what we call a model. The essential question is not that these theoretical constructions are realistic descriptions of the economy, but that they are able to explain the dynamics observed in the economy. Therefore, it is not possible to reject a model ex ante because it is based on assumptions that we believe not too realistic. Rather, the validations must be based on the usefulness of these models to explain reality, and whether they are more useful than other models.

              Regarding the assumption that the lifetime of economic agents is assumed to be infinite: We know that in reality consumers, firms and governments have finite life. However, in our models and to be more precise, we assume that firms and governments both use the infinite time as \textbf{the reference period for taking economic decisions}. This is not unrealistic: no government thinks it will cease to exist at some point in the future and no entrepreneur takes decisions based on the idea that the firm will go bankrupts sometime in the future. For consumers this is not so realistic, however, we may weaken this assumption, and think about families, dynasties or households rather than consumers, then the infinite time planning horizon assumption is feasible. Of course, to study the finite life cycle of an agent (school-work-retirement), the so-called Overlapping Generations (OLG) framework is more useful.
          \end{solution}
\end{enumerate}
\newpage

