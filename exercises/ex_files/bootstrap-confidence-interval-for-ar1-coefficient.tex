Consider the AR(1) model with constant
\begin{equation*}
    y_{t}=c +\phi y_{t-1}+u_{t}
\end{equation*}
for $t=1,\ldots ,T$ with i.i.d. error terms $u_{t}$ and $E(u_{t}|y_{t-1})=0$.
Usually, we construct a $95\%$-confidence interval for $\phi$ using the normal approximation
\begin{equation*}
    \left[ \hat{\phi}-z_{\alpha/2}\cdot SE(\hat{\phi});\ \hat{\phi}+z_{1-\alpha/2}\cdot SE(\hat{\phi})\right]
\end{equation*}
with $\hat{\phi}$ denoting the OLS estimate, $SE(\hat{\phi})$ the estimated standard error of $\phi$ and $z_{\alpha/2}$ the $\alpha/2$ quantile of the standard normal distribution. If one does not know the asymptotic distribution of a test statistic (or it has a very complicated form), one often relies on a nonparametric approach. To this end, we are going to \enquote{bootstrap}, i.e. recompute the t-statistics a large number of times on artificial data generated from resampled residuals.\\
We will do this step-by-step, i.e. write a program for the following:
\begin{itemize}
    \item Simulate $T=100$ observations with $c=1$, $\phi=0.8$ and errors drawn from e.g. the exponential distribution such that $E(u_t)=0$.
    \item Estimate the model with OLS and calculate the t-statistic $\tau=\frac{\hat{\phi}}{SE(\hat{\phi})}$.
    \item Store the OLS residuals in a vector $\hat{u} = (\hat{u}_{2},\ldots ,\hat{u}_{T})'$.
    \item Set $B=10000$ and initialize the output vector $\tau^{\ast} = (\tau_1^\ast,...,\tau_B^\ast)$. \item For $b=1,...,B$:
          \begin{itemize}
              \item Draw a sample \textbf{with replacement} from $\hat{u}$ and save it as $u^{\ast} = u_{2}^{\ast},\ldots ,u_{T}^{\ast }$.
              \item Initialize an artificial time series $y_t^\ast$ with $T$ observations and set $y_1^\ast = y_1$.
              \item For $t=2,\ldots ,T$ generate
                    \begin{equation*}
                        y_{t}^{\ast }=\hat{c}+\hat{\phi}y^\ast_{t-1}+u_{t}^{\ast }
                    \end{equation*}
              \item On this artificial dataset estimate an AR(1) model. Denote the estimated OLS coefficient $\phi^\ast$ and corresponding estimated standard deviation $SE(\phi^\ast)$. Store the following t-statistic in your output vector at position $b$:
                    $$\tau^\ast = \frac{\phi^\ast - \hat{\phi}}{SE(\phi^\ast)}$$
          \end{itemize}
    \item Sort the output vector such that $\tau_{(1)}^\ast \leq ... \leq \tau_{(B)}^\ast$.
    \item  The bootstrap approximate confidence interval for $\phi$ is then
          \begin{equation*}
              \left[ \hat{\phi}-\tau_{((1-\alpha /2)B)}^{\ast }\cdot SE(\hat{\phi});\ \hat{\phi}-\tau_{((\alpha/2)B)}^{\ast }\cdot SE(\hat{\phi})\right]
          \end{equation*}
          Set $\alpha=0.05$ and compare this with the normal approximation.
    \item Redo the exercise for $T=30$ and $T=10000$. Comment on your findings.
\end{itemize}
\begin{solution}\textbf{Solution to \nameref{ex:BootstrapCI}}
    \lstinputlisting{../progs/BootstraptCIAR1.m}
    For large $T$ the bootstrap CI are almost identical to the asymptotic CIs, for small $T$ they are narrower.
\end{solution}
