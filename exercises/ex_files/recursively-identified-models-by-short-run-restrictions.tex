A popular argument in macroeconomics has been that oil price shocks in particular may act as domestic supply shocks for the U.S. economy. Thus, the question of how oil price shocks affect U.S. real GDP and inflation has a long tradition in macroeconomics. Postulate a VAR(4) model with intercept for the percent changes in the real WTI price of crude oil ($\Delta rpoil_t$), the U.S. GDP deflator inflation rate ($\Delta p_t$), and U.S. real GDP growth ($\Delta gdp_t$). Consider the dataset given in the sheet \enquote{USOil} in the file \texttt{data.xlsx}. The data are quarterly and the estimation period is 1987q1-2013q2.

\begin{enumerate}
    \item How can you identify the oil price shock statistically and economically?
          %The model is identified recursively with the real price of oil ordered first such that the real price of oil is predetermined with respect to the U.S. economy, i.e. $y_t = (\Delta rpoil_t, \Delta p_t, \Delta gdp_t)$

          %We focus on the effect of an unanticipated increase in the real price of oil on the real price of oil, on the change in inflation, and on U.S. real GDP growth. The model ispartially identified in that only the oil price shock can be given an economicinterpretation.

    \item Estimate the reduced-form vector autoregressive model by least-squares, maximum-likelihood or Bayesian methods to obtain a consistent estimate of the reduced-form error covariance matrix.

    \item Estimate the structural impact multiplier matrix $B_0^{-1}$ based on a lower-triangular Cholesky decomposition of the residual covariance matrix.

    \item Estimate the structural impact multiplier matrix $B_0^{-1}$ based on solving the system of nonlinear equations that implicitly defines the elements of $B_0^{-1}$ using a nonlinear equation solver that finds the vector $x$ such that $F(x)=0$, where $F(x)$ denotes a system of nonlinear equations in $x$. To this end, vectorize the system of equations $B_0^{-1} B_{0}^{-1'} - \hat{\Sigma}_u$. The objective is to find the unknown elements of $B_0^{-1}$ such that
          $$ F_{SR}(B_0^{-1}) = \begin{bmatrix} vech\left(B_0^{-1} B_0^{-1'} - \hat{\Sigma}_u\right),&b_0^{12},&b_0^{13},&b_0^{23}\end{bmatrix}'=0$$
          where the vech operator is used to select the set of unique elements of $B_0^{-1}B_0^{-1'} - \hat{\Sigma_u}$. To this end:
          \begin{itemize}
              \item Write a function \texttt{fSR.m} that computes $F_{SR}$.

              \item Set the termination tolerance \texttt{TolX=1e-4}, the termination tolerance on the function value \texttt{TolFun} to \texttt{1e-9}, the maximum number of function evaluations \texttt{MaxFunEvals} to \texttt{1e50000}, the maximum number of iterations \texttt{MaxIter} to \texttt{1000}. Save this set of options:\texttt{options = optimset('TolX',TolX,\\'TolFun',TolFun,'MaxFunEvals',MaxFunEvals,'MaxIter',MaxIter)}

              \item Choose an admissible start value for $B_0^{-1}$ and call the optimization routine fsolve to minimize your \texttt{fSR} function:\\	\texttt{[B0inv,fval,exitflag,output]=fsolve('fSR',B0inv0,options,'additional inputs to fSR')}

              \item Impose the normalization rule that the diagonal elements of $B_0^{-1}$ are positive.
          \end{itemize}

          Compare this to the Cholesky decomposition. Note that this approach also works for non-recursively identified models.

    \item Plot the impulse response function of an unexpected increase in the real price of crude oil (not its percent change!) using \texttt{IRFs.m} and interpret it economically.
\end{enumerate}