Consider the VAR(p) model with constant written in the more compact form
$$ y_t = [c, A_1, ..., A_p] Z_{t-1} + u_t = A Z_{t-1}+ u_t$$
where $Z_{t-1}=(1,y_{t-1}',...,y_{t-p}')'$ and $u_t$ is assumed to be iid white noise with nonsingular covariance matrix $\Sigma_u$. Given a sample of size $T$, $y_1,...,y_T$, and $p$ presample vectors, $y_{-p+1},...,y_{0}$, ordinary least squares for each equation separately results in efficient estimators. The OLS estimator is
$$\hat{A} = \left[ \hat{c}, \hat{A_1},...,\hat{A_p}\right] = \left(\sum_{t=1}^{T}y_t Z_{t-1}'\right)\left(\sum_{t=1}^{T}Z_{t-1}Z_{t-1}'\right)^{-1} = Y Z'(ZZ')^{-1}$$
where $Y=[y_1,...,y_T]$ and $Z=[Z_0,...,Z_{T-1}]$. More precisely, stacking the columns of $A = [c, A_1,...,A_p]$ in the vector $\alpha = vec(A)$,
$$\sqrt{T}\left(\hat{\alpha} - \alpha \right) \overset{d}{\rightarrow} \mathcal{N}(0, \Sigma_{\hat{\alpha}})$$
where $\Sigma_{\hat{\alpha}} = plim(\frac{1}{T}ZZ')^{-1}\otimes \Sigma_u$, if the process is stable. Under fairly general assumptions this estimator has an asymptotic normal distribution. A sufficient condition for the consistency and asymptotic normality of $\hat{A}$ would be that $u_t$ is a continuous iid random variable with four finite moments. A consistent estimator of the innovation covariance matrix $\Sigma_u$ is, for example,
$$\hat{\Sigma}_u = \frac{\hat{U}\hat{U}'}{T-Kp-1}$$
where $\hat{U} = Y - \hat{A}Z$ are the OLS residuals. Thus, in large samples,
$$vec(\hat{A}) \overset{a}{\sim}\mathcal{N}(vec(A),(ZZ')^{-1}\otimes \hat{\Sigma}_u)$$
where $\overset{a}{\sim}$  denotes the approximate large-sample distribution. In other words,
asymptotically the usual t-statistics can be used for testing restrictions on individual coefficients and for setting up confidence intervals.
\begin{enumerate}
    \item What are the dimensions of $y_t$, $Y$, $u_t$, $U$, $c$, $A_1$, ..., $A_p$, $A$, $\alpha$, $Z_{t-1}$, $Z$, $\Sigma_u$ and $\Sigma_{\hat{\alpha}}$.
          \begin{solution}
              $y_t$ is $K \times 1$, $Y$ is $K \times T$, $u_t$ is $K \times 1$, $U$ is $K \times T$, $c$ is $K \times 1$, $A_1$ is $K \times K$, ..., $A_p$ is $K \times K$, $A$ is $K \times (1+pK)$, $\alpha$ is $(pK^2+K) \times 1$, $Z_{t-1}$ is $(1+pK) \times T$, $Z$ is $(Kp+1) \times T$, $\Sigma_u$ is $K \times K$ and $\Sigma_{\hat{\alpha}}$ is $(pK^2+K) \times (pK^2+K)$.
          \end{solution}
    \item Consider data for $y_t = (\Delta gnp_t,i_t,\Delta p_t)'$, where $gnp_t$ denotes the log of U.S. real GNP, $p_t$ the corresponding GNP deflator in logs, and $i_t$ the federal funds rate, averaged by quarter. The estimation period is restricted to 1954q4-2007q4. Load the data from  \texttt{data.xlsx} and visualize it. Comment whether you think the data looks stationary.
          \begin{solution}
              The data for Federal Funds Rate as well as the GNP Deflator Inflation does seem to have some trend in it.
          \end{solution}
    \item Estimate a VAR(4) model using the \texttt{VARReducedForm.m} function. This is basically a modified version of your \texttt{ARpOLS.m} function. Examine the stability of the estimated process and the significance of the estimated parameters using t-ratios at a $95\%$ level.
          \begin{solution}
              \lstinputlisting{../progs/ThreeVariableVAR.m}
          \end{solution}
\end{enumerate}
