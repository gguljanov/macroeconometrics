Consider the basic Real Business Cycle (RBC) model with leisure and home production, that is time devoted, for instance, to maintain the house, parenting or nursing care of the elderly. The representative household maximizes present as well as expected future utility
\begin{align*}
    \underset{\{C_{m,t},C_{h,t},I_{t},L_{m,t},L_{h,t},K_{t}\}}{\max} E_t \sum_{j=0}^{\infty} \beta^{j} U_{t+j}
\end{align*}
with $\beta <1$ denoting the discount factor and $E_t$ is expectation given information at time $t$. Total consumption $C_t$ is composed by the consumption of market goods $C_{m,t}$ and of home production services $C_{h,t}$. The aggregation follows a CES-type function:
$$C_t = \left[\omega C_{m,t}^\eta +(1-\omega)C_{h,t}^\eta\right]^{1/\eta}$$
where $\eta$ is the parameter measuring the willingness of agents to substitute between the two goods and $\omega$ is the proportion of each good in total consumption.

The production function for home activities is labor-intensive and given by
$$C_{h,t}=B_t L_{h,t}^\theta$$
where $0<\theta<1$ implies decreasing returns. Productivity $B_t$ is the driving force of the home production sector and evolves according to
\begin{align*}
    \ln{B_{t}} & = \rho_B \ln{B_{t-1}}  + \varepsilon_t^B
\end{align*}
where $\rho_B$ denotes the persistence parameter and $\varepsilon_t^B$ is assumed to be normally distributed with mean zero and variance $\sigma_B^2$.

Non-leisure time $L_t=L_{m,t}+L_{h,t}$ is devoted either to working in the good market production ($L_{m,t}$) or providing home production services ($L_{h,t}$). Hence, the contemporaneous utility function is given by
\begin{align*}
    U_t = \gamma \ln(C_t) + (1-\gamma) \ln{(1-L_{m,t}-L_{h,t})}
\end{align*}
The marginal utility of consumption is positive, whereas more labor in either sector reduces utility. Accordingly, $\gamma$ is the elasticity of substitution between consumption and labor. In each period the household takes the real wage $W_t$ as given and supplies perfectly elastic labor service to the representative firm in the good production sector. In return, she receives real labor income in the amount of $W_t L_{m,t}$ and, additionally, profits $\Pi_t$ from the firm as well as revenue from lending capital $K_{t-1}$ in the previous period at interest rate $R_t$ to the firms, as it is assumed that the firm and capital stock are owned by the household. Income and wealth are used to finance consumption of market goods $C_{m,t}$ and investment $I_t$. In total, this defines the (real) budget constraint of the household:
\begin{align*}
    C_{t,m} + I_t = W_t L_{t,m} + R_t K_{t-1} + \Pi_t
\end{align*}

The law of motion for capital $K_t$ at the end of period $t$ is given by
\begin{align*}
    K_{t} = (1-\delta)K_{t-1} + I_t
\end{align*}
where $\delta$ is the depreciation rate. Assume that the transversality condition is full-filled.

Real profits $\Pi_t$ of the representative firm in the goods market sector are revenues from selling output $Y_t$ minus costs from labor $W_t L_{m,t}$ and renting capital $R_t K_{t-1}$:
\begin{align*}
    \Pi_t = Y_{t} - W_{t} L_{m,t} - R_{t} K_{t-1}
\end{align*}
The representative firm maximizes expected profits
\begin{align*}
    \underset{\{L_{m,t},K_{t-1}\}}{\max} E_t \sum_{j=0}^{\infty} \beta^j Q_{t+j}\Pi_{t+j}
\end{align*}
subject to a Cobb-Douglas production function
\begin{align*}
    Y_t = A_t K_{t-1}^\alpha L_{m,t}^{1-\alpha}
\end{align*}
The discount factor takes into account that firms are owned by the household, i.e. $\beta^j Q_{t+j}$ is the present value of a unit of consumption in period $t+j$ or, respectively, the marginal utility of an additional unit of profit; therefore $Q_{t+j}=\frac{\partial U_{t+j}/\partial C_{t+j}}{\partial U_{t}/\partial C_{t}}$. Productivity $A_t$ is the driving force of the goods market sector and evolves according to
\begin{align*}
    \ln{A_{t}} & = \rho_A \ln{A_{t-1}}  + \varepsilon_t^A
\end{align*}
where $\bar{A}$ denotes the technology level in steady state, $\rho_A$ the persistence parameter and $\varepsilon_t^A$ is assumed to be normally distributed with mean zero and variance $\sigma_A^2$.

Finally, we have the non-negativity constraints	$K_t \geq0$, $C_t \geq 0$, $0\leq L_{h,t} \leq 1$ and $0\leq L_{m,t} \leq 1$ and clearing of the labor as well as goods market in equilibrium, i.e.
\begin{align*}
    Y_t = C_{m,t} + I_t
\end{align*}

\begin{enumerate}
    \item Briefly provide intuition behind the introduction of the home production sector.
    \item Show that the first-order conditions of the household are given by
          \begin{align*}
              E_t\left[\beta \frac{ C_{m,t+1}^{\eta-1}}{C_{t+1}}\left(R_{t+1}+1-\delta \right) \right] & = \frac{C_{m,t}^{\eta-1}}{C_t}                                 \\
              \frac{1-\gamma}{1-L_{m,t}-L_{h,t}}                                                       & = \gamma \omega \frac{ C_{m,t}^{\eta-1}}{C_t} W_t              \\
              \frac{1-\gamma}{1-L_{m,t}-L_{h,t}}                                                       & = \gamma (1-\omega) \theta \frac{ C_{h,t}^{\eta}}{C_t L_{h,t}}
          \end{align*}
          Interpret these equations in economic terms.

    \item Show that the first-order conditions of the representative firm in the goods market sector are given by
          \begin{align*}
              W_t & = (1-\alpha) A_t \left(\frac{K_{t-1}}{L_{m,t}}\right)^\alpha, \\
              R_t & = \alpha A_t \left(\frac{L_{m,t}}{K_{t-1}}\right)^{1-\alpha}
          \end{align*}
          Interpret these equations in economic terms.
    \item Discuss how to calibrate the parameters $\eta$, $\omega$, $\theta$, $\rho_B$ and $\sigma_B^2$.
    \item Write a DYNARE mod file for this model with a feasible calibration and compute the steady state of the model either analytically or numerically.
    \item Study the effects of a positive aggregate technology shock in the goods market sector using an impulse response analysis. Compare this to a model without a production sector, i.e. $\omega=1$.
    \item Simulate data for labor and consumption growth in the goods market for 200 periods. Estimate three parameters (of your choosing) with
          \begin{itemize}
              \item[(i)] maximum likelihood methods
              \item[(ii)] Bayesian methods
          \end{itemize}
          Provide feasible upper and lower bounds and discuss the intuition behind your priors.
    \item Explain whether or not you are satisfied with your estimation results?
\end{enumerate}


%% Homotopy, Macro Language, flip parameters, IRF Matching, GMM, SMM, Projection Methods, Rios-Rull et al source of identification
\newpage
