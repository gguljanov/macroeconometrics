Consider the AR(1) model with constant
$$ y_t = c + \phi y_{t-1} + u_t$$
Assume that the error terms $u_t$ are i.i.d. Laplace distributed with known density
\begin{align*}
    f_{u_{t}}(u)=\frac{1}{2}\exp \left( -|u|\right)
\end{align*}
Note that $E(u_t)=0$ and $Var(u_t)=2$.

\begin{solution}\textbf{Solution to \nameref{ex:MLARpLaPlace}}\end{solution}

\begin{enumerate}
    \item Derive the log-likelihood function conditional on the first observation.

          \begin{solution}
              Computation of the conditional expectation and variance:
              \begin{itemize}
                  \item $E[y_{t}|y_{t-1}] =
                            c + \phi y_{t-1} $,
                  \item $Var[y_{t}|y_{t-1}] = var(u_t) = 2$
              \end{itemize}
              Hence the conditional density is
              \begin{displaymath}
                  f_t(y_{t}|y_{t-1}; c, \phi) = \frac{1}{2} \cdot e^{-|y_{t} -(c + \phi y_{t-1})|} = \frac{1}{2} \cdot e^{-|u_t|}
              \end{displaymath}
              The conditional log-likelihood function is therefore given by
              \begin{eqnarray*}
                  \log L(y_{2}, \dots, y_{T};c, \phi)
                  =-(T-1) \cdot log(2) - \sum_{t=2}^{T} |u_{t}|
              \end{eqnarray*}
          \end{solution}

    \item Write a function that calculates the conditional log-likelihood of $c$ and $\phi$.

          \begin{solution}~
              \lstinputlisting{../progs/LogLikeARpNorm.m}
          \end{solution}

    \item Load the dataset given in the excel file \texttt{Laplace.xlsx}. Numerically find the maximum likelihood estimates of $c$ and $\phi$ by minimizing the negative conditional log-likelihood function.

          \begin{solution}~
              %\lstinputlisting{../progs/ARpMLLaPlace.m}
              %\lstinputlisting{../progs/AR1MLLaPlace.m}
          \end{solution}

    \item Compare your results with the maximum likelihood estimate under the assumption of Gaussianity. That is, redo the estimation by minimizing the negative Gaussian log-likelihood function.

          \begin{solution}
              Maximizing the Gaussian likelihood, even though the underlying distribution is not Gaussian, is also known as pseudo-maximum likelihood or quasi-maximum likelihood. Note that the values are very close to each other.
          \end{solution}
\end{enumerate}


